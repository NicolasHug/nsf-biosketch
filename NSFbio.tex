
%%%%%%%%% BIOGRAPHICAL SKETCH -- 2 pages

\required{Biographical Sketch: Andreas C. M\"uller}
\section*{Professional Preparation}
\begin{tabular}{l l l l}
University of Bonn& Germany& Mathematics     &  Vordiplom 2005\\
University of Bonn& Germany& Mathematics     &  Diplom 2009\\
University of Bonn& Germany& Computer Science&  PhD 2014\\
\end{tabular}

\section*{Appointments}
\begin{tabular}{l l}
Since 2016& Lecturer in Data Science, Columbia University\\
2014--2016& Research Engineer, NYU Center for Data Science\\
2013--2014& Machine Learning Scientist, Amazon Germany\\
\end{tabular}

\section*{Five related products}
\begin{itemize}
    \item \fullcite{mueller2017introduction}
    \item \fullcite{olivier_grisel_2016_49910}  % sklearn package
    \item \fullcite{buitinck2013api}            % api paper
    \item \fullcite{andreas_mueller_2016_49909} % pystruct package
    \item \fullcite{Varoquaux_2015}             
\end{itemize}
\pagebreak
\section*{Five other significant products}
\begin{itemize}
    \item \fullcite{abraham2014machine}
    \item \fullcite{muller2014pystruct}
    \item \fullcite{muller2012information}
    \item \fullcite{scherer2010evaluation}
    \item \fullcite{muller2014learning}
\end{itemize}

\section*{Synergistic Activities}
\begin{itemize}
\item Software Carpentry instructor, contributor to software carpentry and data
    carpentry teaching material.
\item Regular tutorials on machine learning and scikit-learn, materials published under CC-0 license.
\item Contributions to the OpenML open souce project.
\item Contributions to the nbconvert open source project for publishing using Jupyter Notebooks.
\item Organizer of regular ``coding sprints'' to broaden the contributor base of open source projects.
\end{itemize}
