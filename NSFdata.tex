
% \documentclass[11pt]{article}

% \usepackage{amsmath}
% \usepackage{amssymb}



% \setlength{\oddsidemargin}{0in}
% \setlength{\evensidemargin}{0in}
% \setlength{\textwidth}{6.5in}
% \setlength{\textheight}{8.75in}
% \setlength{\topmargin}{-.5in}



% \pagestyle{empty}
% %>\input harvmac

% %\font\Large = cmr12 scaled \magstep 4
% %\font\small = cmr6 scaled \magstep 1
% %\font\large = cmr8 scaled \magstep 2
% %%\font\curlit = cmti10 scaled \magstep 1
% %\font\curs = cmti10 scaled \magstep 2



%\begin{document}

\section{Roles and responsibilities}

\textbf{Principle Investigator: Dr.\ Andreas Mueller}\\
PI Mueller will oversee the data management, and ensure digital preservation of
all produced data. Dr.\ Mueller will also review the continuous use of version
control, as well as the documentation of all experimental procedures and the
documentation of all software artifacts.\\\\
\textbf{Research Engineer:}\\
The Research Engineer will use the github version control platform to
continuously and incrementaly keep track of software development. While some
part of the created software will immediately be integrated in to the
scikit-learn project, other parts of the software will exist in a seperate
github repository. The Research Engineer will also document all software and
experimental protocols within the github projects using the sphinx
documentation generator. The Research Engineer will implement autmatic
compilation of the documentation into a website that will be publicly
available.

Large scale machine learning experiments will be performed on public datasets
hosted on the OpenML platform.
The research engineer will retrieve the data via the OpenML python API, perform
the experiments, and upload the experimental results to the OpenML platform.
The research engineer will also upload the collective results of all
experiments periodically to the figshare platform.

\section{Types of data}

The main output of this project will be python code, which will be archived and
shared via github. Documentation of experimental procedures and the source code
will be contained in the same github repository as the code.

The project will make use of the public machine learning datasets hosted on the
OpenML platform. We will use hundereds of the datasets hosted there, which will
be downloaded via the Python API of OpenML\@. The datasets are stored as ARFF
files, with the metadata stored as JSON\@. The Python API represents these
together as Python objects.

This data will be processed using a variety of algorithms from the scikit-learn
machine learning library, as well as methods that we will implement as part of
the project. The result of the processing will be predictions made by machine
learning algorithms and their accuracy and other evaluation metrics. The
metadata of these results consists of the original dataset, machine learning
task, machine learning model that was used, and the parameters of the model.

These results and the meta-data will be captured and uploaded by the OpenML
Python API to the OpenML platform, which will store it as JSON\@. We will also
separately store the collected results across many datasets and algorithms on
figshare as JSON file. This will provide an  simple access mechanism as
alternative to the OpenML interface for accessing the results. OpenML has
programming interfaces in Java, R, Python, C\# and other languages that will
allow processing of the results.


\section{Policies for access and sharing}

Source code for all software will be made available as an ongoing process
during development. All code and accompanying documentation will be licensed
under a BSD license.

Results will be pushed to OpenML immediately as part of the evaluation,
whenever possible. The results will be archived and published to figshare after
inital analysis confirmed them to be correct.
All data will be made available under the CC-0 license.

\section{Data storage and preservation of access}

The code will be part of an open source project and thereby handed to the open
source community. The initial archive will be github, but the community is
expected to curate and archive the project beyond the livetime of github.
The results will be stored in the OpenML platform and on figshare, both of
which have long-term preservation guarantees.

