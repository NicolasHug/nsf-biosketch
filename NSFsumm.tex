\required{Project Summary}
The open-source machine learning library scikit-learn has become a cornerstone of
applied machine learning and data science in academia and industrial applications.
The PI has been involved in the scikit-learn project as co-maintainer for 5 years.
This project will provide additional resources to maintain and improve the
scikit-learn package.
While widely-used the scikit-learn package has few full-time contributors, and
could benefit from additional engineers dedicated to maintenance and bug fixes.

While the scikit-learn project has received wide recognition for its ease of
use and extensive documentation, many areas for improvement remain.
Scikit-learn was developed with trained machine learning researchers in mind,
but was later adopted by researchers across many disciplines.
We identified two main barriers to entry for domain scientists wanting to apply
machine learning:
The representation of the data needed to apply machine learning, and the choice
of algorithm for a particular dataset and task.

A large amount of research has recently been performed into automatic model
selection, and systematic evaluation of machine learning system.
However, advanced in these areas have not made the transition from cutting edge
computer science research to being applied by domain scientists to solve practical
problems.
By providing guidance on model selection and data preprocessing via
systematic evaluation, and integrating robust implementations of automatic
algorithms selection in the established scikit-learn ecosystem, this proposal
will allow a more effective use of machine learning methods by scientists
without machine learning expertise across fields.

\required{Intellectual Merit}
This proposal advances knowledge in two ways:
\begin{enumerate}
    \item creating a practical implementation of automatic algorithm selection in a
    widely used system will provide a litmus test for current research in automatic
    parameter selection and meta-learning. Exposing current research to a wide user base
    will provide insights into the practicality and usefulness of many recent approaches
    in a way that single studies can not.
    \item lowering the barrier of entry for applying machine learning even further,
    and improving the existing tools provided by scikit-learn, will enable more
    researchers to adopt machine-learning solutions to data-driven problems.
\end{enumerate}

\required{Broader Impacts}
Machine learning has become a core part of many data driven research projects,
as well as commercial services. While applying machine learning used to be
a privilege of companies and institutes that can afford large computational
teams or expensive commercial software and services, the rise of open
source tools for data analysis has leveled the playing field considerably.

In particular the python ecosystem of data science tools has been widely
adopted in the scientific community, both for research and in teaching.
As a core component of the data science environment, maintaining and
improving scikit-learn will benefit all researcher and teaching relying
on this ecosystem of tools.

Outside of the academic environment, scikit-learn has had a tremendous
impact in allowing small companies as well as established tech giants
to easily incorporate machine learning in their products and services.

With the enhancements described in this proposal, even more people
will be able to easily apply machine learning to their problems,
without requiring large amounts of machine learning training.

% There are 4 kinds of broader impacts.
% 1. advance discovery and understanding while promoting teaching,
% training and learning
% 2. broaden the participation of underrepresented groups
% 3. disseminated broadly to enhance scientific and technological
% understanding
% 4. benefits of the proposed activity to society
